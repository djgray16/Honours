\section{Outline}
This literature review provides the necessary knowledge to construct agent-based models on random graphs. It is divided into five sections that build up to the current state of the literature. Section \ref{CGT} defines and demonstrates classical game theory, as well as the deficiencies of classical models. Section \ref{EGT} introduces evolutionary game theory and discusses its relationship to the classical models. In Section \ref{ABM}, evolutionary game theory is extended to agent-based models. This is followed by Section \ref{RG}, which discusses the development of random graphs and their applications as models for social networks. Finally, the current state of the literature is discussed in Section \ref{EGToRG}. 

\section{Classical Game Theory} \label{CGT}

\subsection{Formal Representation}
Game theory is the abstraction of interaction between entities. While there exist earlier examples of game theory, its formal genesis is attributed to von Neumann and Morgenstern in 1944 \cite{RN99, RN27}. The key specifications of a \emph{normal form game} are the \emph{players, $\{i, i = 1, \dots ,n\}$}, \emph{strategy sets $ \{S_i, i = 1, \dots ,n\}$}, with $ S_i = \{s_{1}^i, s_{2}^i,\dots, s_{k_i}^i\}$, where $s_{j}^i$ is the $j$th pure strategy for player $i$. Importantly, players may also play mixed strategies; linear combinations over their pure strategies. A \emph{mixed strategy} $s^i = \sum_{j} \alpha_j^i s_{j}^i$, $~\sum_{j} \alpha_j^i = 1$, $~\alpha_j^i \geq 0$, allows the player to randomly select a strategy $s_{j}^i$ from their strategy set $S_i$ with probability $\alpha_j^i$. It can be viewed as an element of the simplex $\Delta_{S_i}$, whose vertices are the pure strategies,$\{s_{1}^i, s_{2}^i,\dots, s_{k_i}^i\}$.  A \emph{strategy profile}, $\mathbf{s}$, is a tuple of strategies $(s^1, s^2, \dots, s^n)$ which represent the (possibly mixed) strategy each player in the game has chosen. The strategy profile may also be written as $(s^i, \mathbf{s^{-i}})$, where $\mathbf{s^{-i}}$ represents the tuple of strategies chosen by all agents except $i$. The \emph{payoff functions} $\pi_i$, map strategy profiles to payoffs for each player $\pi_i: S_1 \times \cdots \times S_n \to \mathbb R$. In the case of mixed strategies being played, the payoff function is the $\alpha$-weighted payoff of the mixed strategy combinations. \\ % $$\pi_i(s^i, s^{-i}) = \mathbb E [\pi_i(s^i, s^{-i})] = \sum_{s_j^i \in S_i} \alpha_j^i \sum_{k=1, k \neq i}^n \sum_{s_l^k \in S_k} \alpha_l^k \pi_i(s_j^i, s_l^k)$$.% \\


% Games can also be defined as \emph{extensive form games}. In addition to the specification of a normal form game, extensive form games are defined on a tree, where each end node is assigned to a payoff $(\pi_1,\dots,\pi_i,$ In this case, the information set $\mathcal I_i$ must also be defined for each player. For player $i$, $\mathcal I_i $ is a partition of their nodes such that the player cannot distinguish nodes within the same element of the partition. $= \{ \}$\cite{RN78}. In this thesis, all games are of perfection information, which means that every player is aware of the history of moves. In the case of simultaneous move games, they are 

%Also needs nodes, payoff of path, information partition consisting of indistinguishable nodes\\

 Von Neumann and Morgenstern make the distinction between a \emph{game}, the abstract set of players, actions, payoffs, and information sets, and a \emph{play} of the game, consisting of a singular instance of the game, from start to finish. The same distinction is used in this review. \\



The players, also called agents, each have complete knowledge of the rules of the game. At any point in the game, their possible moves are enumerated in $S_i$, which can depend on the state of the game. Finally, when the game ends, they receive a numerical payoff calculated by $\pi_i$. Players are risk-neutral, in that they seek to maximise their expected payoff. Additionally, they are also aware that all other players are also aiming to maximise their respective payoffs.\\

A game is of \emph{perfect information} if, at each point in the game, each player knows the entire history of the game. Backgammon is an example of a game of perfect information. On the other hand, bridge is not, as players know their own cards without knowing other players' \cite{RN78}. \\


\subsection{Example: Prisoner's Dilemma} \label{PD}
Perhaps the most famous game is the Prisoner's Dilemma (PD). It was formalised by Tucker, Nash's PhD supervisor \cite{RN82}. Consider a two-person conspiracy of thieves, each arrested and held separately at the police station. If they both stay silent (cooperate with each other), they will be imprisoned on a lesser charge with one year jail time. However each thief is separately offered the choice to defect to the police, and inform on their co-conspirator. The defector will receive immunity, and the other thief receives five years in prison if they stay silent. Finally, if both thieves defect, their testimony is not as valuable so they are imprisoned for three years each. A matrix representation of the game is in Table \ref{table:PD}. 

\begin{center}
\begin{table}[h]
\centering
\begin{tabular}{|l|l|l|}

    

\hline
          & Cooperate & Defect \\ \hline
Cooperate & (-1,-1)     & (-5,0)  \\ \hline
Defect    & (0,-5)     & (-3,-3)  \\ \hline

\end{tabular}
    

\caption{Matrix Representation of Prisoner's Dilemma.}
\label{table:PD}
\end{table}
\end{center} 
In the matrix representation convention, the first player is represented in the rows. Their payoff from a strategy profile is the first number in the payoff tuple. Vice versa, the second player has a choice of columns, and their payoff is the second element of the tuple. \\

Formally, this game is represented by players $\{1,2\}$, strategy sets $S_1 = S_2 = \{C,D\}$, payoff functions $\pi_1, \pi_2$ defined below, 
\begin{align*}
    \pi_1& : S_1 \times S_2 \to \mathbb R, ~ \pi_2: S_1 \times S_2 \to \mathbb R\\
    \pi_1&(C,C) = -1,~ \pi_1(C,D) = -5,~ \pi_1(D,C) = 0,~ \pi_1(D,D) = -3.\\
    \pi_2&(C,C) = -1,~ \pi_2(C,D) = 0,~ \pi_2(D,C) = -5,~ \pi_2(D,D) = -3.\\
\end{align*}

\subsection{Example: Public Good Game} \label{PGG} \label{ToTC}

The PD game extends to an $n$-player game, initally called \emph{Tragedy of the Commons}, and formalised by Hardin in 1986 but drawing from a sketch in 1833 by Lloyd \cite{RN82}. In modern examples, each player can contribute some of their wealth $c$ to a common pot, where it is amplified by return rate  $r$. Regardless of contribution, all players equally divide the common pot.  \\

This sketch can be normalised as follows. Each player $\{ i = 1,\dots,n\}$ has the strategy set $S_i = \{c_i, c_i \in [0,1] \}$, and payoff function $$\pi_i = -c_i + \frac{\sum_{j =1}^n r c_j}{n}$$ with return ratio $r$ satisfying $n>r>1$. In the literature, this game is called a Public Goods Game (PGG) \cite{RN67}. For further examples of common games, see \cite{RN79}, or Appendix A of \cite{RN99}.  \\

% \textbf{Definition}\\
% Each player has a set of strategies $S_i = \{s_1, s_2,\dots, s_k\}$ which is the set of possible actions that they can take. \\

% A \emph{strategy profile} is a tuple of strategies $(s_1, s_2, \dots, s_n)$ which represent the strategy each player has chosen. \\

% In the example above, the set of strategies  is $\{$ Cooperate, Defect $\}$ for each player. \\

% The payoffs of a game are the reward for a player $i$ as a function of the strategy profile. Formally, the payoff is a function
%  $\pi_i: S_1 \times S_2 \times \dots \times S_i \to \mathbb R$.  \\

% Another famous game is Battle of the Sexes. A man and a woman are at their respective houses, and unable to communicate with each other. The man prefers opera to boxing, and the woman boxing to opera. Ultimately, they always prefer to be together than alone. A matrix representation is shown, with the man represented in the rows, and the woman in the columns. \\

% \begin{table}[h]
% \begin{tabular}{|l|l|l|}
% \hline
%       & Opera & Boxing \\ \hline
% Opera  & (3,1) & (0,0)  \\ \hline
% Boxing & (0,0) & (1,3)  \\ \hline
% \end{tabular}
% \caption{Matrix Representation of Battle of the Sexes
% }
% \label{BotS}
% \end{table}





To further discuss these games, there must be some method of comparing strategy profiles under the assumptions of classical game theory. \\

\subsection{Nash Equilibria}
The first Nobel Memorial Prize in the field of game theory was awarded to John Nash, who developed the concept of a Nash Equlibrium (NE). A NE is a profile of strategies such that no player has a unilateral incentive to deviate from the profile by altering their strategy. \\


A (possibly mixed) strategy profile $\mathbf{s^*} = (s^{*1}, s^{*2}, \dots, s^{*n})$ of a game with $n$ players is a NE if \\
\begin{align}
    \pi_i(\mathbf{s^*}) \geq \pi_i(s^{*1}, s^{*2},\dots, s^{*k-1}, s^{k}, s^{*k+1} \dots, s^{*n}), \qquad \forall  s^{k} \in \Delta_{ S_i}, \forall i =\{1, \dots, n\}. \label{NE}
\end{align}


Using the alternative notation, this can be written as \\
\begin{align*}
    \pi_i(s^{*k},s^{*-k}) \geq \pi_i(s^k,s^{*-k}), \qquad \forall  s^k \in \Delta_{ S_i}, \forall i =\{1, \dots, n\}.
\end{align*}

This is an extension of von Neumann's minimax theorem on zero-sum games \cite{RN99}, and Nash proved the existence of at least one NE under the assumption of a finite set of players, each with a finite set of possible actions, where the player is allowed to play a mixed strategy. \\


\subsubsection{Game with no Nash Equilibrium: Largest Number Game}
It is simple to demonstrate a game with no NE, however it requires that the set of possible actions be infinite. Consider a two-player game, with $S_i = \{s^i: s^i \in [0,1) \}$ and $\pi_i(s^i,s^j) = \mathbbm{1}_{\{s^i>s^j\}}$. This is symmetric, so consider the payoff function for player 1. \\
\begin{align*}
    \pi_1(s^1,s^2) = \mathbbm{1}_{\{s^1>s^2\}}
\end{align*}
If \eqref{NE} holds for player 1, than $1>s^1 >s^2$ and also there exists an $\epsilon>0$ such that $1>s^2 + \epsilon>s^1$. So \\
\begin{align*}
    \pi_2(s^2,s^1) = 0 \qquad \text{and} \qquad \pi^2(s^2 + \epsilon,s^1) = 1.
\end{align*}
It is evident that \eqref{NE} cannot hold for player 2, as they should move to $s^2 + \epsilon$ to improve their payoff. This is shown for arbitrary $s^1, s^2$, so holds for all $s^1 \in [0,1), s^2 \in [0,1)$. Hence this game does not have a NE. \\

\subsubsection{Pareto Optimal}
An alternate solution concept is Pareto optimal strategy profiles. A strategy profile $\bs^*$ \emph{Pareto dominates} another strategy profile $\bs$ if $$\forall i=1,\dots,n, \quad \pi_i(\bs^*)\geq \pi_i(\bs), \quad  \text{and} \quad  \exists  j\in [1,\dots,n] \quad \text{such that} \quad \pi_j(\bs^*)>\pi_j(\bs).$$ In other words, a Pareto dominated strategy profile can be improved to another strategy profile such that no agent's payoff decreases, and at least one agent's payoff increases \cite{RN97}. A strategy profile $\bs^*$ is \emph{Pareto optimal} if there does not exist any strategy profile that dominates $\bs^*$. A NE is not necessarily Pareto optimal, and a Pareto optimal strategy profile is not necessarily a NE. Pareto optimal strategy profiles can be considered as the solutions preferred by a \emph{benevolent central planner}, who can unilaterally force each agent to choose a particular strategy. In classical game theory, this central planner does not exist, and agents are selfish, so NE is the logical solution concept. \\


\subsection{Nash Equilibrium and Pareto Optimal Solution to Example \ref{PD}}


It is quite simple to verify that in the PD \eqref{PD}, the NE is $(D, D)$. The game is symmetric with respect to the players, so it suffices to show \eqref{NE} for the first player only. The strategy set for each player is simply $\Delta(C,D) = \{\alpha C + (1-\alpha D)|$   $\alpha \in [0,1] \}$.  For any strategy $s$ in the set $\Delta(C,D)$, the payoff $\pi_1(s,D)$ is given as follows: \\
\begin{align*}
\pi_1(s,D) &= \pi_1(\alpha C + (1-\alpha)D,D) \\
&=\alpha \pi_1(C,D) + (1-\alpha) \pi_1(D,D) \qquad \text{(Payoff of mixed strategy is expectation)} \\
&= \alpha (-5) + (1-\alpha)(-3) = -3 -2\alpha. \\
\end{align*}
For the strategy to be a NE, it must satisfy \eqref{NE}. The strategy corresponding to pure defection, $s^* = D$, satisfies this condition as \\
\begin{align*}
    \pi_1(s^*,D) =\pi_1(D,D) = -3 \geq -3-2\alpha \qquad \forall \qquad \alpha \in [0,1].
\end{align*}
Hence $(D,D)$ is a NE of \eqref{PD}. Note that all of $(C,D), (D,C)$, and $(C,C)$ are  Pareto optimal strategy profiles, while the NE is Pareto dominated by $(C,C)$.  \\


\subsection{Nash Equilibrium and Pareto Optimal Solution to Example \ref{ToTC}}

In the $n$-player extension, the NE is $(0,0,\dots,0)$. Once again, this game is symmetric so it suffices to consider the first player. Their strategy set is $S = \{c, c \in [0,1] \}$. For any strategy $c$ in $S$, the payoff when playing $(c,0,0,\dots,0)$ is \\
\begin{align*}
    \pi_1(c,0,\dots,0) &= -c + \frac{\sum_{j=1}^n rc_j}{n} \\
    &= -c + \frac{rc + 0 + \dots + 0}{n} \\
    &= c\left(-1+\frac{r}{n}\right)<0 \qquad \text{as $1<r<n$.}
\end{align*}
This is decreasing in $c$, so \eqref{NE} holds only for $c=0$. Therefore the NE is $(0,0,\dots,0)$. Once again, the NE is Pareto dominated by $(1,1,\dots,1)$. \\

In general, it is much easier to verify a NE than to find them. Techniques for finding a NE include Brown's fictitious play, iterated dominance, and best-response functions \cite{RN99}. These are discussed further in \cite{RN99}. 
% In Battle of the Sexes, there are three NE. They are (Opera, Opera), (Boxing, Boxing), and the mixed strategy where each players chooses their preferred event $\frac{3}{4}$ of the time. \\
\subsection{Deficiencies of Classical Game Theory} \label{Differences}
 \subsubsection{Perfect Rationality}
A key assumption of classical game theory is rational players , and the common knowledge that all other players are rational, \emph{ad infinitum} \cite{RN78}. This form of rationality is known as perfect rationality \cite{RN99}. Consider a finitely repeated PD game. Each player knows that the last round they will be defected against, so they will play $D$ in the last round. This backward induction repeats for the penultimate round, so the only NE is pure defection. However, this assumption did not correspond to experimental results in some games \cite{RN99, RN50}. Therefore either the model of the game is incorrect, or human players do not fulfil the model assumptions. From this example, the idea of bounded rationality was proposed. Bounded rationality limits the computational power of each agent when deciding their strategy. Examples of boundedly rational strategy rules are given in \ref{UD}.   \\

\subsubsection{Game Dynamics}
Under the assumption of perfect rationality, there are no dynamics to the game, because each player perfectly anticipates all future moves and instantaneously constructs the equilibrium solution. However under a bounded rationality regime, the dynamics of the game become crucial. The model for rationality must consider the agents' memory (knowledge of game history), myope (how long agents think ahead), search strategy, and learning technique. All of these factors influence the dynamics of the game. \\

\subsubsection{Illogical Nash Equilibria}
The definition of a NE allows for equilibria which appear suboptimal. Consider the two-player matrix game in Table \ref{SG}.  \\

\begin{table}[hbt!] %\label{SG}
\begin{center}
\centering
\begin{tabular}{|l|l|l|}
\hline
  & C     & D     \\ \hline
C & (1,1) & (0,0) \\ \hline
D & (0,0) & (0,0) \\ \hline
\end{tabular}
\caption{Simple Game.
}
\label{SG}
\end{center}

\end{table}
Both $(C,C)$ and $(D,D)$ are NE. Yet it is evident that the $(C,C)$ equilibrium should always be preferred. A possible criterion to eliminate the illogical $(D,D)$ NE could be Pareto optimality. However example \ref{PD} demonstrated that the Pareto optimal solution is not always NE. Other examples of \say{illogical} NE are non-credible threats in sequential games such as the market entry game \cite{RN79}. The next section of this review discusses advances in the literature that remedy these flaws. \\


\section{Evolutionary Game Theory} \label{EGT}

Evolutionary game theory (EGT) is a relaxation of the assumption of perfect rationality. The seminal work on EGT was John Maynard Smith's \emph{Evolution and the Theory of Games} \cite{RN99}. In an EGT model, a classical \emph{stage game} is repeated a large number of times $T$. Agents are not necessarily rational, but instead strategies \emph{evolve} according to a game-wide heuristic that simulates evolution. EGT remedies three deficiencies of classical game theory; the unreasonable assumption of perfect rationality, lack of consideration for game dynamics, and lack of equilibria selection in the case of multiple NE \cite{RN99}. For completeness, it should be noted that several authors had proposed refinements of NE before EGT, such as Bayesian NE, risk dominance, and trembling hand perfection \cite{RN39}. EGT achieves the full potential of these refinements and extends them further.   \\

 
% The evolutionary model of the mind is defined by behavioral rules. The population may consist of heterogeneous agents. The strategy evolves as plays of the game endow agents with fitness, which then determines their evolutionary success \cite{RN99}.   \\
\subsection{Population Games}
In the early development of evolutionary game theory, games were constructed such that contemporary solution techniques could be applied. Games defined in this manner will be referred to as \emph{population}, or \emph{mean-field} games. \\

 Population games make several key assumptions to allow the behaviour of the population to be investigated analytically. These are: \begin{itemize}
    \item Common payoff function $\pi$,  $\pi_i = \pi$ for all agents $i = 1, \dots, N$. 
    \item Very large number of boundedly rational agents $N \to \infty$.
    \item Few classes $C = \{1,\dots, M\}$ of agents, where agents with common class are homogeneous (have the same strategy set). A game with exactly one class is called \emph{symmetric}. 
    \item Equal probabilities of matching with any other agent in each stage of the game. 
    \item A common strategy update rule, that is applied rarely relative to the number of plays, so the rule applies to the average success of a strategy. It is this update rule that simulates evolution.
    %\item Small discount factor $\delta$ \textbf{TO CHECK}.
    
\end{itemize}



Just as Nash introduced the NE to characterise plausible solutions of a one-stage game, John Maynard Smith introduced the concept of an evolutionary stable strategy (ESS) for symmetric population games. 
\subsection{Evolutionary Stable Strategy}
In the context of symmetric population games, a strategy $s^* \in \Delta_S$ is an ESS if, for all other possible strategies $s \in \Delta_S$, the strategy $s^*$ is resistant to invasion from $s$. A \emph{pure} ESS is an ESS consisting of a pure strategy, that is, a strategy in the corner of the simplex $\Delta_S$. Mathematically, an ESS  $s^*$ satisfies  \\
\begin{align}
    \pi(s^*,s^*) >& \pi(s, s^*) \qquad \forall s \neq s^* \qquad \text{or} \label{ESS1}\\
    \pi(s^*, s^*) =& \pi(s, s^*) \qquad \text{     and      } \qquad \pi(s^*, s) > \pi(s,s) \qquad \forall s \neq s^*. \label{ESS2} 
\end{align}
Note an alternative definition is provided by \cite{RN80}. They instead consider an $\epsilon$ proportion of the population playing $s \neq s^*$. Then the population is playing $\overline{s} = (1-\epsilon)s^* + \epsilon s $. A strategy is an evolutionary stable strategy if: \\
\begin{align*}
    \pi(s^*, \overline{s}) > \pi(s, \overline{s})
\end{align*}
for all $\epsilon$ sufficiently close to 0 and $s \neq s^*$. \\

Condition \eqref{ESS1} is often called the \emph{Strict NE condition}, and \eqref{ESS2} is called \emph{Maynard Smith's second condition}. Evidently, every ESS is a NE, however the converse is not true. In the Simple Game, Table \ref{SG}, both $(C,C)$ and $(D,D)$ are NE. Condition \eqref{ESS1} is satisfied for $C$, because \\
\begin{align*}
    \pi(C,C) = 1 > \pi(D,C) = 0. \\
\end{align*}
On the other hand, $D$ is not an ESS because neither \eqref{ESS1} or \eqref{ESS2} are satisfied. \\
\begin{align*}
    \pi(D,D) &= 0 \ngtr \pi(D,C) = 0, \text{ and}\\
    \pi(D,D) &= \pi(D,C) \qquad \text{but } \pi(D,C) \ngtr \pi(C,C).
\end{align*}
In this game, the ESS solution set does not include the \say{illogical} solution $(D,D)$. This example demonstrates the improvement of ESS over NE. \\


Another example bimatrix game, called chicken, admits two pure NE but no pure ESS. The game involves two players either daring $(D)$ or ceding $(C)$ to challenge for a common goal. The game derives its name from the adolescent game whereby two drivers line up opposite each other on a straight road. They drive towards each other at speed in a game of brinksmanship, and the first person to cede is derided for being a chicken. Both drivers daring results in a dangerous head-on collision with highly negative payoffs. Both drivers being chicken results in a neutral stalemate, while one ceding driver loses social standing relative to the daring driver. The value of the social standing is much less than the loss due to a crash. \\

\FloatBarrier
\begin{table}[hbt!]
\begin{center}
\centering
\begin{tabular}{|l|l|l|}
\hline
  & C      & D         \\ \hline
C & (0,0)  & (-1,1)    \\ \hline
D & (1,-1) & (-20,-20) \\ \hline
\end{tabular}

\caption{Chicken. This game is an analogue for nuclear brinksmanship. It was posited in the evolutionary context that animals could either use threat displays like a dove $(C)$, or physical attacks, like a hawk $(D)$, in contesting a shared resource. The problem formulation requires that each actor commit to their choice before seeing the other's move.
}
\label{Chicken}
\end{center}
\end{table}
\FloatBarrier
The pure NE are $(C,D)$, and $(D,C)$. The strategy set is $\Delta(C,D) = \{\alpha C + (1-\alpha) D|$,   $\alpha \in [0,1] \}$. This game is symmetric, so condition \eqref{NE} is checked for player 1 only. \\
\begin{align*}
    \pi_1(s,D) &= \alpha\pi_1(C,D) + (1-\alpha)\pi_1(D,D) \\
    &= \alpha (-1) + (1-\alpha)(-20) = -20 + 19\alpha. 
\end{align*}
This is increasing in $\alpha$, so \eqref{NE} only holds for $\alpha = 1$, which corresponds to always playing $C$. The same formulation can be used for the NE $(D,C)$. \\
\begin{align*}
        \pi_1(s,C) &= \alpha\pi_1(C,C) + (1-\alpha)\pi_1(D,C) \\
    &= \alpha (0) + (1-\alpha)(1) = 1-\alpha. 
\end{align*}
This is decreasing in $\alpha$, so $\eqref{NE}$ only holds for $\alpha = 0$, which is playing $D$ always. Finally, there is a mixed equilibrium where each player plays $C$ with proportion $\alpha^* = 0.95$. To prove this strategy profile is a NE, first note that any mixed equilibria, denoted $( s^i, s^{-i})$ must satisfy the condition that the agent $i$ is indifferent between their pure strategy profiles. \\
\subsubsection{Lemma: Mixed Strategy Equilibrium}
\emph{In a mixed strategy NE $ = (s_i,s_{-i})$, then each pure strategy being played with positive probability, $s_j^i$ such that $\alpha_j^i>0$, must have the same payoff against the mixed strategy profile $s_{-i}$.  }
\subsubsection{Proof}
 Without loss of generality, suppose player 1 is playing a mixed strategy $\alpha^* \in (0,1)$. Their opponents are playing $\mathbf{s^{-i}}$, also mixed. For contradiction's sake, suppose there exist pure strategies $s_{j_1}^i, s_{j_2}^i$ such that $\alpha_{j_1}^i, \alpha_{j_2}^i >0$ and $\pi_i(s_{j_1}^i,\mathbf{s^{-i}}) > \pi_i(s_{j_2}^i,\mathbf{s^{-i}})$. Then the strategy modification of playing $s_{j_1}^i$ with probability $\alpha_{j_1}^i + \alpha_{j_2}^i$, and not playing $s_{j_2}^i$ yields a higher payoff. This is a contradiction, as the original strategy was a NE and satisfied \eqref{NE}. Hence any mixed strategy NE $(s_i,s_{-i})$ requires that $\pi_i(s_j^i, \mathbf{s^{-i}})$ is constant for all $s_j^i$ with $\alpha_j^i>0$. \halmos \\

Now to solve for the mixed equilibrum of Table \ref{Chicken}. Once again, it suffices to check \eqref{NE} for player 1. From the lemma, the mixed strategy equilibrium requires $\pi_1(C,s^{-i}) = \pi_1(D,s^{-i})$. 
Consider player 2 playing any $\alpha \in [0,1]$. \\
\begin{align*}
    \pi_1(C,\alpha C + (1-\alpha)D) &= \alpha (0) + (1-\alpha)(-1) = \alpha -1 \\
    \pi_1(D,\alpha C + (1-\alpha)D) &= \alpha (1) + (1-\alpha)(-20) = -20 + 21\alpha\\
    \pi_1(C,\alpha C + (1-\alpha)D) &= \pi_1(D,\alpha C
    + (1-\alpha)D) \implies \alpha = 0.95
\end{align*}
Therefore the unique mixed equilibrium is each player playing $C$ with probability 0.95. \\


In an evolutionary game context, this game is called Hawk--Dove, as introduced by Maynard Smith. The pure strategies $C,D$, are not ESS, as neither \eqref{ESS1} nor \eqref{ESS2} hold. The only ESS is the mixed NE demonstrated above, because it satisfies \eqref{ESS2}. The first equation of \eqref{ESS2} is satisfied by the working above. To satisfy the second equation of \eqref{ESS2}, consider any mixed strategy $\beta \neq 0.95$.  \\
\begin{align*}
    \pi_1(\alpha &= 0.95,\beta) = \alpha \beta (0) + \alpha(1-\beta)(-1)+ (1-\alpha)\beta(1) + (1-\alpha)(1-\beta)(-20) \\
    &= -1.95 + 2\beta\\
    \pi_1(\beta,\beta) &= \beta^2(0) + \beta(1-\beta)(-1) + \beta(1-\beta)(1) + (1-\beta)^2(-20)\\
    &= -20(1-\beta)^2
\end{align*}
As $2-1.95\beta> -20(1-\beta)^2$ for all $\beta \neq 0.95$, then condition \eqref{ESS2} holds and the mixed strategy is an ESS. In fact, a mixed ESS can be viewed as a proportion of the population playing each pure strategy.   \\



\subsection{Replicator Dynamics}
An evolutionary game requires a strategy update rule. Depending on the model, this may be stochastic, and a variety of rules have been formulated. Taylor and Jonker, 1978, introduced \emph{replicator dynamics}, and it is the pre-eminent update rule for biological models. Given a homogenous population with infinite size and $|S| = n$ ($n$ possible strategies), playing a strategy profile $s$, with proportion $\rho_i$ playing strategy $s_i$, then the rate of change of that proportion is given by: \\
\begin{align} \label{Replicator}
    \dot{\rho_i} \propto \rho_i(\pi(s_i, s) - \pi(s, s) ) \qquad\forall i = 1, \dots, n.
\end{align}


\subsection{Example: Axelrod's Iterated PD} \label{IPD}
An early example of a population game came from Axelrod's computer tournaments \cite{RN89}. In the late 1970's, political scientist Robert Axelrod invited contestants to submit strategies to play the iterated PD. The first round received 14 submissions, and the second round received 62 entries from across six countries, including John Maynard Smith (who came 24th). Axelrod then extended the second round into a mean-field game, simulating 1000 generations. Each entry was paired with every other entry, as well as a clone of itself and a strategy playing purely randomly. The length of the game was chosen such that the probability of it ending after a given play was 0.00346, with an expected median length of 200. Strategies were allowed to consider the entire history of plays up to the current round. \\

One NE of the iterated PD is to always defect. Another NE of the iterated PD is the so-called grim trigger (GT), that cooperates until any player defects, and then always defects. For games expected to be sufficiently long, this is a NE. Of course, some participants submitted strategies that were not NE. For example, tit-for-tat (TFT) initially plays $C$, then copies the opponent's previous move. B\H{o}rufsen (BF) plays TFT except for a check for 3 consecutive alternating moves, in which case it reverts to $C$ and continues. For further specification of the strategies, refer to \cite{RN89}. Axelrod used replicator dynamics to simulate the evolution of these strategies over 1000 generations, shown in Figure \ref{Axelrod}. Strategies are labelled according to their rank in the tournament. \\
\FloatBarrier
\graphCap{axelrod.png}{1}{Axelrod's Population Game, from \cite{RN89}. Rule 1 is TFT, Rule 3 is BF.}{Axelrod}
\FloatBarrier

Although TFT won all three rounds of Axelrod's tournament, later authors have proven that it is not an ESS, in the sense that a population playing purely TFT can be successfully invaded by another strategy \cite{RN90}.
\subsection{Folk Theorem of EGT}
The \emph{Folk Theorem of Evolutionary Game Theory} links ESS and NE concepts. It is called the Folk Theorem because it was not published until it was allegedly already widely known. To apply the folk theorem of EGT, one more definition is needed. A strategy profile $\mathbf{s}$ is a rest point if the RHS of \ref{Replicator} is 0. The folk theorem states:

\begin{itemize}
    \item If $\mathbf{s}$ is a NE, then it is a rest point. If $\mathbf{s}$ is a strict NE, then it is an attractor. 
    \item If a solution to \eqref{Replicator} is an orbit that converges to $\mathbf{s}$, then $\mathbf{s}$ is a NE. 
    \item If a rest point is stable, then it is a NE. Stable here is the stability of the ODE $\eqref{Replicator}$. 
\end{itemize}
For further characterisations and proof, see \cite{RN99, RN81}. The folk theorem can be viewed as a sequence of refinements of equilibria. It says,  for a given game, the set of ESS is a subset of the set of symmetric NE, and the set of strict NE is a subset of the ESS \cite{RN91}.  \\


In some cases, replicator dynamics can be solved analytically using mean-field theory. In fact, the orbits solved by the replicator equation \eqref{Replicator} can be transformed into orbits of the Lotka--Volterra equation below with the map $y_i = \frac{\rho_i}{\rho_n}$ \cite{RN81}\\
\begin{align*} \label{LV}
    \dot{y_i} = y_i\left ( r_i + \sum_{j =1}^N c_{ij} y_j \right ),
\end{align*}
where the constants have been normalised as follows: $r_i = \pi(s_i,s_n) - \pi(s_n,s_n)$ and $c_{ij} = \pi(s_i,s_j) - \pi(s_n,s_j)$. \\


This theorem allows the mathematical analysis of Lotka--Volterra models to be applied to game theory. With the exception of \cite{RN75}, population games do not consider the network structure that dictates the interaction between agents. Population games assume either a complete graph \cite{RN67}, or random uniform sampling across the agents \cite{RN47}. The 2008 paper by Dunia \cite{RN75} demonstrates the link between stochastic dominance of a graph and mean-field approximations on the \emph{tragedy of the commons} game. This will be expanded upon in chapter \ref{MFT}. \\

Mean-field theory fails to capture certain key elements of evolutionary game theory. Firstly, the population size is often finite. Secondly, population games neglect the network on which agents interact. In the last 20 years, this has become a crucial specification of a model. \\

Mean-field theory makes several strict assumptions about an evolutionary game. The assumption of infinite agents is one, as is the implicit assumption of a complete interaction graph. These assumptions can be relaxed, however analytical solution techniques can no longer be applied. This leaves agent-based modelling as the major solution technique. \\ 


\section{Agent-Based Modelling} \label{ABM}

An \emph{agent-based} game is an evolutionary game defined at a microscopic level. It is defined by: 

\begin{itemize}
    \item Stage game played a large number of times $T$.
    \item  Common payoff function $\pi$,  $\pi_i = \pi$ for all agents $i = 1,\dots, N$. 
    \item Large but finite number of agents $N$ interacting on a graph $G$. They play the stage game with their graphical neighbours.
    \item Few classes $C = \{1,\dots, M\}$ of agents, where agents with common class are homogeneous (have the same strategy set). 
    \item A common strategy update rule, that is applied rarely relative to the number of plays, so the rule applies to the average success of a strategy.
    %\item Small discount factor $\delta$ \textbf{TO CHECK}.
\end{itemize}

\subsection{Update Dynamics} \label{UD}
Agent-based games allow for more specification of the update dynamics. As well as the replicator dynamics that generally characterise population games, alternate strategy update rules, collated from \cite{RN43}, are \emph{imitation of the best}, in which a small fraction of the population recreate the strategy of the highest ranked agent, \emph{local replicator dynamics}, which is replicator dynamics restricted to a neighbourhood of each agent, and the \emph{Fermi rule}. Imitation dynamics are often used for the growth of cultural phenomenon, as opposed to biological growth rates where replicator dynamics are standard \cite{RN30}. The \emph{Fermi rule}, from \cite{RN43}, is explained momentarily. It considers agent $i$ playing $s_i$ and a randomly chosen neighbour playing $s_j$. The transition probability is defined as \\
\begin{align*}
    \mathbb P(s_i \to s_j) = \frac{1}{1+\exp(-\beta(\pi_j - \pi_i))},
\end{align*}
where $\beta$ is a chosen parameter. In a physics based-interpretation, this $\beta^{-1}$ represents the temperature. Strategy updates are more error-prone as the temperature is increased. \\

An esoteric population update rule is a conformity chasing agent, modelled in \cite{RN23}, where a certain subset of the agents emulate the modal strategy. Another update method, from \cite{RN81} is best response (BR) dynamics. In general, this assumes more cognition on behalf of the agents. At each time-step, a small proportion of the population update their strategy $s^i$ such that 
\begin{align*}
    s^i \in \textrm{BR}(s), \qquad \textrm{BR}(s) = \{s^*| \pi(s^*,s) \geq \pi(s_j,s), \qquad \forall s_j \in S\}.
\end{align*}

\subsubsection{Example: Imitation Dynamics} \label{Lottery}
To iterate the importance of the specified update dynamics, Nau and Roos \cite{RN30} developed the following example of a 2-stage lottery game. Each play consists of two lotteries, where agents can choose to play $S$, the safe lottery with guaranteed reward $4$, or $R$, the risky lottery with $1-p$ chance of $0$, and $p$ chance of $8$. Importantly, they are allowed to condition their second lottery on the result of the first. This leads to 6 pure strategies, so $C = \{\textrm{SS, RR, SR, RS, R--WR, R--WS} \}$. 
\begin{itemize}
    \item SS: safe both times. 
    \item RR: risky both times. 
    \item SR: safe in the first round, risky in the second. 
    \item RS: risky in the first round, safe in the second.
    \item R--WR: risky in the first round, and if it is a win then risky again, otherwise safe.
    \item R--WS: risky in the first round, and if it is a win then safe, otherwise risky.
\end{itemize}
Obviously these strategies all have the same expected value for $p=0.5$. Under replicator dynamics, the expected growth rate $\mathbb E[\pi_i - \pi_j] = 0$ for all strategies $i,j$, so the initial equilibrium generally persists. This is supported by simulation \cite{RN30}. However under imitation of the best, the growth rate of a strategy is proportional to its probability of beating another strategy, rather than its expectation. Denote $W_i$ the event that the $i$th lottery is won, and $L_i$ its complement. So then,\\
\begin{align*}
    \mathbb P(\textrm{R--WS}> \textrm{SS}) &= \mathbb P(W_1) = 0.5, \\
    \mathbb P(\textrm{SS}>\textrm{R--WS}) &= \mathbb P(L_1 \cap L_2) = 0.25, \\
    \mathbb P(\textrm{R--WS} = \textrm{SS}) &= \mathbb P(L_1 \cap W_2) = 0.25.\\
\end{align*}
So R--WS is imitated twice as often as SS under pairwise comparison (in the case of a tie, the winner is chosen by a coin flip). The other combinations are not computed here, but one can verify that SS, RR, and R--WR should go extinct. This was verified under simulation, and is plotted in Figure \ref{Nau} from \cite{RN24}, which was a conference presentation before the paper \cite{RN30}. 
\graphCap{Nau.png}{1}{Imitation Dynamics of Two--Stage Lottery Game, from \cite{RN24}.}{Nau} 

In the paper \cite{RN30}, the authors proved that R--WS is an ESS. In fact, for the risky lottery with $p \neq 0.5$, R--WS is an ESS against the expected value maximising strategy for a range of $p$. This example is replicated in Chapter \ref{Lottery_Me}. \\


As early as 1984, Axelrod described an agent-based model for playing the repeated PD on a grid \cite{RN63}. Each agent plays with their von Neumann neighbours. The update rule considered by these agents was imitation of the best strategy amongst an agent's neighbours. This model was extended by Nowak and May, 1992 and defines a cellular automaton, \cite{RN63}, \cite{RN70}. \\



 Huberman and Glance, 1993, claimed that a spatial grid was not a plausible model \cite{RN63}, because it relied on a global clock, known to all agents, to update their strategies collectively. In the case where strategies were updated \emph{asynchronously} --- iteratively picking an individual at random and then updating their strategy, the long-run equilibria was always defection. \\
 
 Fortunately, this was only true in a subset of the game parameter space. The field of EGT rapidly started to expand. \cite{RN45} considered expanding from pairwise 2-player games to 3-player games on a hexagonal lattice, where each agent hosts one game with all neighbours, and also plays in games they are invited to. They also introduced the opportunity to punish defection. \cite{RN66} introduced the idea of a loner strategy, whereby any game involving a loner is cancelled and each player receives a small payoff. In  \cite{RN67}, the loner strategy was considered for a well-mixed population, and extended to regular lattices in \cite{RN66}. \\
 
 Around the same time, Barab\'{a}si and Albert published \emph{Emergence of Scaling in Random Networks} \cite{RN55} and Watts and Strogatz published \emph{Collective Dynamics of \say{Small-World} Networks} \cite{RN58}. These papers were to revolutionise models of social networks, and agent-based models were not immune. The next section explores the literature of random graphs as models for social networks. \\
 
 \section{Random Graphs as Models for Social Networks} \label{RG}
%  The modern study of agent-based modelling requires an underlying graph structure with agents living at nodes. The next section examines the literature of random graphs as models for social networks.\\
 
 \subsection{Erd\H{o}s--R\'enyi Graphs}
 The first random graph was proposed by Erd\H{o}s and R\'enyi at the end of the 1950s \cite{RN53}. In fact, two alternate models of a random graph were suggested. Firstly, they considered a set of vertices, enumerated $V = \{1,2,\dots,n\}$ and a fixed number of edges $M$. Then all possible graphs with $n$ vertices and $M$ edges were enumerated, and the random graph is chosen uniform randomly from this collection. This is called a $G(n,M)$ model. The second formulation, denoted $G(n,p)$ also requires the vertex set $V$, but edges are created independently. It is also called the Erd\H{o}s--R\'enyi--Gilbert model, after Gilbert who proposed this model independently \cite{RN64}. A parameter $p$ is given, and each vertex pair $(i,j)$ is independently connected by an edge with probability $p$ \cite{RN56}. The expectation for the number of edges in the $G(n,p)$ graph, $\X [E]$, is \\
 \begin{align*}
     \X [E] = p {n \choose 2} = p\frac{n(n-1)}{2}.
 \end{align*}The average degree, $\X [K]$ can be estimated as \\
 \begin{align*}
     \X [K] = \frac{2}{n} \X [E] = p(n-1).
 \end{align*}
 
 As early as 1967, Davis, Leinhardt and others \cite{RN64} realised that real-world networks exhibited different characteristics to those observed in Erd\H{o}s--R\'enyi graphs. A triad, a complete subgraph of three nodes, is much more common in observed social networks than in Erd\H{o}s--R\'enyi graphs. Key characteristics of a graph are the degree distribution, $P(k) = \Pm [K=k]$, clustering coefficient, $C_\Delta$, and average shortest path length, $L$ \cite{RN53}. These are enumerated in the later section \ref{CGC}. \\
 
 Famously, one's Erd\H{o}s and Bacon numbers are indicators of low $L$ in real-world citation and filmography networks respectively. Erd\H{o}s and Reyni did not seek to model real-world networks, but mathematicians attempting to do so needed to develop more realistic models of real-world networks.\\
 
 \subsection{Exponential Random Graph Model}
 The exponential random graph model, alternatively called the $p^*$ model, was coined by Wasserman and Pattison, in honour of Leinhardt and Holland, who developed a model with one explanatory variable, a $p_1$ model \cite {RN83}. The $p^*$ model conceptualises the observed network $\by$ as one realisation of a set of possible networks generated by an underlying random process that depends on explanatory variables $\bz(\by) = [z_1(\by), z_2(\by), \dots, z_r(\by)]$ \cite{RN64}. The explanatory variables may be graph-theoretic measurements, such as the total number of edges, triangle proportion, or they may be external measurements of the actors at nodes \cite{RN86}. By assumption, the number of nodes is fixed, so it is not a parameter \cite{RN64}. Akin to generalised linear models \cite{RN83}, the underlying stochastic process is controlled by parameters $\bp = [p_1,p_2,\dots,p_r]$ that are coefficients of the linear function $$\sum_{i=1}^r z_i(\by)p_i. $$ 

 
 Assume that each edge $Y_{ij}$ is a random variable. The edges can be written into an $n\times n$ random matrix, denoted $\bY$. A particular realisation is denoted $\by$. Then the probability of observing the network $\by$ can be written as \\
 $$ \Prob (\bY = \by) = \frac{1}{D}\exp{\sum_{i=1}^r z_i(\by)p_i},$$
 
 where $D$ is a normalising constant. To avoid the intractability of $D$, the log odds are considered and conditioned on all other observed edges $y_{ij}^c$. This is the pseudo-log-likelihood. Write 
 $$\omega_{ij} = \log{\frac{\Prob (Y_{ij} =1 | y_{ij}^c)}{\Prob (Y_{ij} =0 | y_{ij}^c)}}.$$ 
 
$p^*$ models assume that these log-odds are conditionally independent. Denote $y_{ij}^+$ and $y_{ij}^-$ as the observed networks when edge $(i,j)$ is forced to be 1 and 0 respectively. Then the explanatory variables of the network are denoted $\bz(y_{ij}^+), \bz(y_{ij}^-) $ when $(i,j)$ is fixed at 1 and 0 respectively. This is simply computing the value of the explanatory variables for both cases of edge $(i,j)$. Then the pseudo-log-likelihood can be written $$
 \omega_{ij} = \exp \big (\bp \cdot [\bz(y_{ij}^+) - \bz(y_{ij}^-)] \big).$$
This is computationally tractable, and $\bp$ minimises the product of the $\omega_{ij}$ over every possible link in the graph. Actual solution techniques are discussed in Wasserman and Pattison's 1997 paper, but were found by Strauss (1986) and Strauss and Ikeda (1990) \cite{RN83}. Of note, Monte Carlo methods are demonstrably useful for parameter estimation \cite{RN86}. \\

A common explanatory variable is the triangle count, $\tau = \sum_{i,j,k} X_{i,j}X_{j,k}X_{k,i}$, so a $p^*$ model of a social network generally has a triangle parameter $p_\tau$ \cite{RN84}. Supposing that the researchers assumed the age of the actor has an effect as well, the simplest modelling technique is to introduce a different triangle parameter for each age class $(p_{\tau_{a_1}}, \dots, p_{\tau_{a_k}})$ for $k$ different age classes \cite{RN64}. Once a model has been developed, random graphs can be sampled from the specified distribution $\bp$ for analysis. For agent-based modelling, creating random graphs from a distribution with known parameters $\bp$ is useful to test how agent-based phenomenon interact on different graphs \cite{RN86}. \\

This model for networks has boundless potential for parametrisation. However there are two more common base models for social networks. The first is small-world networks, which seek to rectify the low clustering coefficient of $G(n,p)$ graphs to more accurately model social networks. \\
 
 

 \subsection{Small-World Networks}
In 1998, Watts and Strogatz sought to model real-world networks more accurately than $G(n,p)$ graphs. They noticed that $G(n,p)$ graphs do not generate local clustering. Therefore their proposed model, designated here a \emph{small-world} model, seeks to achieve a more accurate clustering coefficient \cite{RN58}. The term small-world refers to the fact that the average shortest path length $L$ is very low relative to the number of nodes $N$. This was expounded in Watts' popular science book, \say{Six Degrees: The Science of a Connected Age} which popularised the Bacon number \cite{RN57}.   \\

The algorithm to construct a small-world graph $\mathrm{WS}(n,\kappa,\beta)$ starts with a regular ring lattice with $n$ nodes, even mean degree $\kappa$ and each node connected to exactly $\kappa$ closest neighbours. Then for every node $i$, proceed to rewire each edge $j$ with probability $\beta$. That is, with probability $1-\beta$, it remains unaltered, and with probability $\beta$ the edge is deleted and replaced with an edge connecting $i$ and $l$, with $l$ chosen randomly amongst all nodes except for $i$ and $j$ \cite{RN58}. \\

If $n \gg \kappa \gg \ln(n) \gg 1 $, then the graph will almost certainly be connected. Watts and Strogatz observed that the clustering coefficient only falls with high $\beta$, and this produces the small-world phenomenon. These characteristics are shown in Figure \ref{SWG}, from \cite{RN93}. In this case, the graph is parameterised by $p$ instead of $\beta$, and the characteristics are scaled by their theoretical maximum. The graph shows that there is a possible selection for $p$ with relatively high clustering coefficient and low average shortest path length.  \\
\FloatBarrier
\graphCap{SWG.png}{1}{Average Shortest Path Length and Clustering Coefficient of Small-World Graphs, parametrised by $p$ and scaled. From \cite{RN93}}{SWG}
\FloatBarrier
In the paper, the authors modelled the spread of an infectious disease on a small-world graph relative to a $G(n,p)$ graph, and demonstrated that the average number of time-steps to infect the whole graph decreased. As an extension, they also modelled the iterated PD on the graph, and used TFT as well as a mutant. They found the average cooperation level decreasing in $\beta$ \cite{RN58}. \\

Small-world graphs accurately modelled the clustering coefficient of real-world networks, however they did not follow the degree distribution of real-world networks. In fact, their degree distribution is the Dirac-delta function centered at $\kappa$, which creates a homogenous network, as opposed to the power-law distribution of real-world networks. Only a year later, Barab\'{a}si and Albert used preferential attachment to model the degree distribution of real-world networks. \\
 
 
 
 
 
\subsection{Preferential Attachment Model} \label{BA}
While Watts--Strogatz rectified the low clustering coefficient of Erd\H{o}s--R\'enyi graphs, Barab\'{a}si and Albert developed a graph model that mimics the degree distribution of real-world graphs \cite{RN55}. Real-world graphs exhibit a scale-free power law degree distribution. Some examples are citation links and pages on the internet \cite{RN55}. According to \cite{RN56}, the original paper used erroneous data for their model of the internet, and in fact underestimated the likelihood of high-degree nodes on the internet. \\

There are two keys to a Barab\'{a}si--Albert $\mathrm{BA}(n,m)$ model. They are preferential attachment, and network growth. The algorithm begins with $n_0$ fully connected nodes, and new nodes are added iteratively. Each added node is connected to $m<n_0$ nodes, with a probability proportional to the degree of each node, $K_i$. The probability that the node being added is connected to existing node $i$, $p_i$ is \\
\begin{align*}
    p_i = \frac{K_i}{\sum_j K_j},
\end{align*}
where the sum is taken over all pre-existing nodes $j$. Remarkable to the BA model is that it seems to model the \emph{process} in which real-world networks are created, as well as the result \cite{RN55}. However the BA model does not perfectly represent real-world networks. In the generating process, the degree of an added node is fixed at $m$, and therefore each node has degree of at least $m$. This is an artifical minimum degree which does not correspond to real-world networks. \\

Barab\'{a}si, along with Bianconi, \cite{RN59} later published a paper that considers the fitness of a node, $\eta_i$, which can represent any attractive quality. Then the probability a new node is connected to node $i$ is \\
\begin{align*}
    p_i = \frac{\eta_ik_i}{\sum_j \eta_jk_j}.
\end{align*}
This is an example of the fitter-gets-richer phenomenon that quantitatively models the evolution of competitive systems. 

\subsection{Extensions}
Since the development of the BA model, the idea of growth models has been explored. Antonioni and Tomassini \cite{RN51} used a growth model in two-dimensional unitary space to attain a higher clustering coefficient. The $n_0$ nodes are generated randomly in $[0.45,0.55]^2$, and then each additional node is generated randomly in $[0,1]^2$. Each additional node is connected to an existing node according to the BA algorithm with probability $\alpha$, or added to its closest Euclidean neighbour with probability $(1-\alpha)$. This generates a higher clustering coefficient than the original BA model \cite{RN51}.  An $\alpha$ of 0.1 is a good approximation for empirically observed networks. \\  

Gabel and Redner \cite{RN60} questioned how an incoming node can \emph{see} the degree distribution of the whole graph to correctly \emph{calculate} the probabilities. Instead, they propose that each incoming node is randomly assigned to a target node, and it joins that target node with probability $1-r$, otherwise it joins the parent of the target node. This local growth model with fixed $r$ is equivalent to the BA model. If $r$ is a decreasing function of the parent's node, then sublinear preferential attachment is achieved, which is also found in empirical studies. Finally, they prove that superlinear preferential attachment is not possible under local redirection. This is evidence for the hypothesis that real-world graphs grow with agents only having local knowledge. Furthermore, it is more efficient to simulate network growth \cite{RN60}. 



\subsection{Common Graph Characteristics} \label{CGC}

There are three key characteristics of a graph that will be discussed. They are the degree distribution $\mathbb P(K_i = k)$, clustering coefficient $C_\Delta$, and average shortest path length $(L)$. The degree distribution is the probability that a randomly chosen node $i$ has degree $k$. The clustering coefficient $C_\Delta = n^{-1}\sum_{i=1}^n C_{\Delta_i}$, where each $C_{\Delta_i}$ is calculated as $$
 C_{\Delta_i} = \frac{\hat{E_i}}{K_i(K_i-1)}$$ where $\hat{E_i}$ is the number of edges amongst neighbours of $i$. Finally, the average shortest path length is the average of every pair of nodes' minimum distance $D(i,j)$. This is $$D(i,j) = \min_{\text{paths  } i \to j} |\text{path}|,$$ $$L = {n\choose 2} ^{-1}\sum_{i=1, j=1, i \neq j}^{(n,n)} D(i,j). $$ The key results for $G(n,p)$, $\mathrm{WS}(n,\kappa,\beta)$, and $\mathrm{BA}(n,m)$ are summarised. Note that exponential random graph models have too broad of a parametrisation space to specify. Also, \emph{small-world} graphs are split into two cases, as $\beta \to 0$ and $\beta \to 1$ respectively. 
 
 
 \begin{table}[hbt!]
\begin{center}
\centering
\begin{tabular}{|l|l|l|l|}
\hline
   & $\mathbb P(K=k)$      & $C_\Delta$ & $L$         \\ \hline
$G(n,p), \mathrm{WS}(n,\kappa,\beta \to 1)$ & ${{n-1}\choose k} p^k (1-p)^{n-1-k}$  & $C_\Delta = \frac{\kappa}{n-1}$  &  $ L\propto \frac{\ln n}{\ln \kappa}$ \\ \hline
$\mathrm{WS}(n,\kappa,\beta \to 0)$ & $\delta_\kappa$& $C_\Delta \to \frac{3(\kappa-2)}{4(\kappa-1)}$& $L \propto \frac{n}{2\kappa}$  \\ \hline 
%$WS(n,K,\beta \to 1)$ & TBA&  $C \to \frac{K}{n-1}$ &  $ l\propto \frac{\ln n}{\ln K}$ \\ \hline
$\mathrm{BA}(n,m)$ & $\frac{2m(m+1)}{k(k+1)(k+2)} \propto k^{-3}$ & $C_\Delta = \frac{\ln(n)^2}{n}$ &$\lim_{n \to \infty}L \to \frac{\ln (n)}{\ln(\ln(n))} $ \\ \hline
\end{tabular}
\end{center}
\caption{Common Graph Characteristics.
}
\label{CGCG}
\end{table}

Results for the $G(n,p)$, $\mathrm{WS}(n,\kappa,\beta)$, and $\mathrm{BA}(n,m)$ graphs come from \cite{RN53, RN58, RN94, RN95, RN96}. 

 
 \section{Evolutionary Game Theory on Random Graphs} \label{EGToRG}
 
 The intersection of random graphs and EGT has produced many recent papers. The following summaries represent a cross-section of EGT results.  \\
 
 \subsection{Optional Public Good Game on Varying Networks} 
 In 2002, two papers by \cite{RN47, RN67} used mean-field theory as well as Monte Carlo simulations to model an optional PGG on a well-mixed population. An optional PGG allows players to contribute $c=1$, defect $c=0$, and play as a loner with $\pi_{3} = \sigma<r-1$. A random sample of size $N$ is selected, and the PGG is played with all non-loner players. This is an adaption of the standard PGG described in \ref{PGG}. The paper \cite{RN47} used replicator dynamics, as in \ref{Replicator}. Obviously, the corners of the simplex $\{e_c, e_d, e_l\}$, where $e_c, e_d, e_l$ correspond to the proportion of the population playing cooperate, defect, loner, respectively,  are unstable fixed points. In the notation of \eqref{Replicator}, these proportions would be denoted $\rho_c,\rho_d,\rho_l$. Furthermore, the replicator equations are transformed into a Hamiltonian system and a heteroclinical cycle (rock-paper-scissors) is observed on the boundary. For $r\leq 2$, the only fixed points are the corners of the simplex. This is demonstrated in Figure \ref{cycle1}. \\
 \graphCap{cycle1.png}{0.8}{Optional PGG, $ N =5,r=1.8, \sigma = 0.5$, from \cite{RN47}. The corners of the simplex correspond to the whole population playing $c,d,l$ respectively. One can see that there is no interior fixed point, and all orbits converge to $e_l$. However $e_l$ is not stable, as one agent playing $c$ will receive higher payoff. } {cycle1}
 
 
 For $r>2$, there is exactly one interior fixed point in the simplex. \\
 \FloatBarrier
 \graphCap{cycle2.png}{0.8}{Optional PGG, $N = 5, r= 3, \sigma = 1$, from \cite{RN47}. In this case, the interior fixed point $Q$ is neutrally stable and surrounded by closed orbits. In fact, $Q$ can be positioned anywhere in the simplex by controlling the parameters $N,r,\sigma$. }{cycle2}
 \FloatBarrier
 The mean-field theory is supported by Monte Carlo simulations in \cite{RN67}, first on a well-mixed population and then extended to a regular lattice. The regular lattice demonstrated higher proportions of cooperation for a given $r$. This is evidence for the hypothesis that graph structure can influence the cooperation level of a game. This paper initially considers replicator dynamics, and then imitation and best-response. Under best-response dynamics, the oscillations of the cycle were damped, as agents can move faster to a local equilibrium. This is shown in Figure \ref{cycle3}. \\
 
 \FloatBarrier
 \graphCap{cycle3.png}{1.4}{Optional PGG played with best response dynamics, from \cite{RN47}. $N=5, r=3, \sigma = 1$ \cite{RN67}. Best response dynamics allow for \emph{smarter} agents, and the interior stable point becomes an attractor. The dashed lines delineate the areas where each strategy dominates. }{cycle3}
 \FloatBarrier
 
 
 The dependence on the social network is also noted in papers such as \cite{RN65}. It compared exponential and scale-free networks and was extended to small-world networks. \\
 
 \subsection{Efficiency-based Replicator Dynamics on BA Graphs}
 Weber and Porto \cite{RN77} applied a discrete version of replicator dynamics \eqref{Replicator} to BA graphs. However the standard assumption of accumulated payoffs is invalid on BA graphs, as the equations of motion are not invariant under affine transformation of the payoff function \cite{RN77}. This is because players have vastly different degrees, so their earnings per round may vary as a result of the degree and not necessarily strategy. Therefore the authors adapt the payoff function to $\hat{\pi_i} = K_i^{-1} \pi_i $ and apply replicator dynamics. Under this model, the \say{hubs} of the network are less able to support cooperation, and overall cooperation decreases. This vastly reduced the difference between scale-free and regular lattice networks in a pairwise NE's dilemma game. However this is only one particular model, and differences in cooperation due to network structure were demonstrated in \cite{RN62, RN44, RN43, RN28}.  \\
 
 
 \subsection{Bipartite Graphs: Separating the Game and Update Networks }
 Pena and Rochat \cite{RN48} argued that bipartite graphs were a better model for PGG on graphs. They model two different variations of an $N$-player Prisoner's Dilemma (NPD). A \emph{conventional} NPD involves a fixed cost $c$ for every game that a cooperator plays in. A \emph{distributed} NPD charges the fixed cost per cooperator, who proportions it equally amongst all their games. As in a PGG, the total contribution is scaled by $r$ and distributed to all players equally. \\
 
 
 A bipartite graph consists of two sets of vertices, top vertices $T$ and bottom vertices $\bot$, as well as one set of undirected edges that connect elements of $T$ with elements of $\bot$. In their model, the bigraph $B$ consists of top vertices that represent games, and bottom vertices that represent players. This paper clearly separates the \emph{interaction} graph $G$ (who plays with who) and the \emph{replacement} graph $H$ (who competes with who). In general, games played on BA graphs assume equality between these two graphs, and the separation reduces overall cooperation, because the replacement neighbourhood is much larger for the hubs.  \\
 
When the replacement graph, $H$, is given by $G$ itself, called (OG) in Figure \ref{bigraph}, this is the traditional unigraph approach. Alternatively, $H$ can be found as the projection of $B$. The projection of $B$ creates a link between players for every game played by these two players. In this model, players who are second neighbours in $G$ may become neighbours in $H$. The \emph{unweighted projection} (UP) model creates a link if there exists a game in common for two players. In contrast, the \emph{unnormalised weighted projection}, (UWP), weights the link according to the number of games in common between two players. Finally, the \emph{normalised weighted projection}, (NWP), normalises the weight of a game according to its size such that $w_{ij} = \sum_k \frac{\delta_i^k\delta_j^k}{n_k}$, where $n_k$ is the number of players in game $k$, and $\delta_i^k$ is 1 if player $i$ plays in game $k$. \\

The results for this model are shown in Figure \ref{bigraph}. \\

\graphCap{bigraph.png}{0.75}{NPD played on bigraphs, from \cite{RN48} A: Conventional NPD played on ring. Because a ring has constant degree $k_i$, the NWP is the same as the UWP. B: Conventional NPD played on BA graph. C: Distributed NPD played on BA graph. Common mean degree of 4. It is evident that the bigraph approaches consistently reduce the observed cooperation in the network. The $x$-axis represents the enhancement factor $\eta$, called $r$ elsewhere in this review. The $y$-axis is the observed cooperation level in the steady state. }{bigraph}
 
 
 
 
 
 \subsection{Evolutionary Games on Evolving Networks}
 Extending the endogeneity of the network even further, recent papers have considered evolving networks as well. For example, Tomassini and Antonioni proposed random fluctuations \cite{RN41,RN42}, migration \cite{RN50, RN62}, and even developed their own adaptation of Barab\'{a}si--Albert graphs that included a spatial dimension \cite{RN51}. Spiekermann \cite{RN68}, Zhang et al.~\cite{RN71} investigated agents that were able to modify their network, and demonstrated improved cooperation. In \cite{RN72}, the population was dispersed on demes (islands), and random migration is proposed between demes. \\
 
 






\subsection{Summary}
This review covers the essential components of classical and evolutionary game theory. It begins with a survey of Classical Game Theory, in Section \ref{CGT}. Section \ref{PD} and \ref{ToTC} are well-known 2- and $n$-player games from the literature. The first solution concept of the literature review; the NE of a game, is defined in Section \ref{NE} and provided for the games mentioned above. A game without a NE is also presented. Pareto optimality is also briefly discussed, before some flaws of classical game theory are mentioned at the end of Section \ref{CGT}.  These flaws are:\\

\begin{itemize}
    \item The assumption of perfect rationality is a flaw, because perfect rationality is not observed in human experiments and models with this assumption do not reflect observed experiments. 
    \item A corollary of the perfect rationality assumption is the lack of game dynamics. Relaxing the assumption leads to analysable game dynamics.
    \item  Finally, the NE solution concept is flawed, and Table \ref{SG} is a game that demonstrates an illogical NE.
    
\end{itemize} 
These flaws lead to the next section, EGT.   \\

EGT is split into two sections; Section \ref{EGT} focusses on \emph{population} games, ESS and the folk theorem of EGT. The advantage of ESS relative to NE is shown through the example games Table \ref{SG} and Table \ref{Chicken}. In Section \ref{IPD}, Axelrod's iterated PD is discussed. Finally, Section \ref{EGT} closes by demonstrating  that NE are a subset of ESS, and ESS are a subset of strict NE. This is called the folk theorem of EGT. \\

EGT is also reviewed from an agent-based modelling perspective. Under this regime, agents are stationed at nodes of a graph and play the game with their graphical neighbours. Varying the graph characteristics can have an effect on the outcome. The advantages of agent-based modelling are the ability to extend models to further update dynamics, and to model agents on random graphs. Section \ref{ABM} gives a tour of alternative update dynamics used in modelling EGT. \\


Section \ref{RG} explores the history and development of random graphs as models for social networks. The two common models are \emph{small-world} and BA models. They are the models for graphs used in EGT on networks, so it is important that their strengths and weaknesses are noted. This discussion occurs in Section \ref{RG}. Their limiting characteristics are presented in Table \ref{CGC}. Some extensions to these graphs are also presented. \\

Finally, a selection of modern papers regarding EGT on random graphs are discussed in the final section \ref{EGToRG}. The common theme of these papers is that the graph structure influences the outcome of a game. In general, the average cooperation level is measured as the dependent variable, and the relationship between average cooperation and game structure, graph structure, and update dynamics are investigated. \\

An important concept from these papers is the extension of the PD to include a \emph{loner} choice. This modified PD is examined under a \emph{population} and also agent-based Monte Carlo lens. \\

Also in this section EGT on bipartite graphs are discussed. This distinction separates the competition and strategy neighbourhood. These discussions provide avenue for further research and demonstrate the current frontier of research. \\




