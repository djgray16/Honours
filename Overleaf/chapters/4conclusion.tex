\section{Key Findings} \textbf{NEEDS BETTER TITLE}
Chapter \ref{Chapter:Lit} defines classical game theory and explores the expansion to evolutionary game theory, and then to evolutionary game theory on graphs. There is a section on random graphs as models for social networks, which informs the choice of random graph models for the simulations. The section \ref{Simple} contains two lemmas which motivate the inclusion of a graph structure for EGT analysis of the linear PGG. Without a graph structure, full defection is the only possible equilibrium, and no further analysis is required. Finally, there is a review of the current literature on PGG on graphs. The reference \cite{RN49} serves as a starting point for simulation in Chapter \ref{TA}. \\

From Chapter \ref{Chapter:Lit}, four graph models are chosen for simulation. These are the WS model, with rewiring $p=0.1$, the TAG model, with proportion parameter $\alpha=0.3$, the BA model, and the RRG model. The value of each parameter is taken from suggestions in the literature. Each model has targeted mean degree $m=6$, and $n=100$ nodes. The true mean degree is a random variable under the TAG and BA model, however Table \ref{graph_stats} shows that the targeted mean degree is well approximated. Each of the WS, TAG, and BA model serve as models for social networks, but are quite different in clustering coefficient $C_\Delta$ and distribution of degree size. The RRG model is included as a control. \\


In Chapter \ref{TA}, the model proposed by Tomassini and Antonioni in \cite{RN49} is successfully replicated and then analysis is extended to examine the robustness of results, as well as the effect of different random graph models. The game is the linear PGG, but the agent model is a payoff satisfaction model, as opposed to an EGT model. The purpose of the payoff satisfaction model is to recreate decisions made by humans in a laboratory environment. Under the payoff satisfaction model, it is found that graph models with high clustering coefficient $C_\Delta$ enforce conformity in contribution, regardless of the distribution of degree size. That is, relative to models with a lower $C_\Delta$, high $C_\Delta$ graph models induce lower cooperation in a low $r$ regime and higher cooperation in a high $r$ regime. This is because agents are more successful in influencing their neighbours, and conformity is the result of high $C_\Delta$. \\

In Chapter \ref{Chapter:Rep}, simulation of replicator dynamics is introduced. Firstly, the paper \cite{RN30} is replicated and extended. The paper involves a lottery game described in Section \ref{Lottery}. The paper was chosen for its simple implementation of replicator dynamics, and it was successfully replicated. This provides authority to the simulations in Chapter \ref{Chapter:Rep} and Chapter \ref{Chapter:ID}. After the lottery paper is investigated, replicator dynamics are transposed to the linear PGG on random graphs. It is found that the random graph models with high degree size variance induce higher cooperation. To explain this phenomenon, several further simulations are carried out. The effect of the rewiring parameter $p$ in the WS model, and $\alpha$ in the TAG model on cooperation are examined. A new model, the PL model, is introduced to examine the effects of $C_\Delta$ in a power-law degree distribution environment. These results indicate that degree size variance, as opposed to $C_\Delta$, is responsible for the difference in cooperation level. A possible explanation is the high proportion of low degree nodes, which are readier to contribute. It is shown that lower mean degree $m$ results in higher cooperation, so it is feasible that graph models with a high proportion of low degree nodes induce higher cooperation.  \\

Chapter \ref{Chapter:ID} compares the graph models under imitation dynamics. Imitation dynamics use the same replicator equation \eqref{rep}, instead with $\alpha=0$. This enforces faster strategy switching, and the time to equilibrium is quicker for imitation than replicator dynamics. The same simulations from Chapter \ref{Chapter:Rep} are run in Chapter \ref{Chapter:ID}, and results are similar. The trend of positive correlation between degree size variance and cooperation is again confirmed. Relative to replicator dynamics, graph models in an imitation dynamics regime induce lower cooperation. This is because the rate at which strategies change is higher, and there is insufficient time for reorganisation of clusters of cooperation which are necessary for the proliferation of cooperation. A major difference between replicator and imitation dynamics is the performance of the RRG model. Under imitation dynamics, it fails to ever achieve non-zero cooperation for $r<m+1$.   \\


 
\section{Outlook}
There is opportunity for further research in this field. As a starting point, the effect of changing $N$ was not considered. This may have an effect on cooperation, and is quite an important avenue for research. The simulations in this model did not consider a metric for measuring whether equilibrium has been achieved. Instead, each simulation was stopped at a fixed $T$. If a metric had been constructed, this would save computational time, and provide more confidence in conclusions. However there is no simple metric that shows equilibrium has been achieved if the equilibrium is non-trivial. \\

The major conclusion from this thesis is the relationship between degree size variance and cooperation. The proffered explanation is the high proportion of low-degree nodes, which are more likely to contribute as the relative value of $r$ is higher. However, the explanation could be strengthened by an investigation into sub-structures of the graph model. For example, how does a fully connected component of size $n$ react when one agent is also connected to a large defecting (or contributing) hub. These sub-structures are not possible to solve analytically, but there is potential for simulation to provide insight. The two lemmas in Section \ref{Simple} are a first attempt at solving for some specific structures, but there are many more to be investigated. \\




