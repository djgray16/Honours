\section{Motivation}
This thesis investigates the linear public goods game, played on a random graph structure, with agents evolving according to evolutionary game dynamics. The linear public goods game is an abstraction of the tragedy of the commons, and is a model for the sharing of a public good. An example of a public goods game is a pot-luck party, where each attendee is invited to bring a plate of food. The dynamics of this game are investigated under evolutionary game theory. Evolutionary game theory is the progression of classical game theory, with less strict assumptions on the rationality of agents. It is used in biological and cultural contexts, to model the proliferation of strategies, and has been used to explain altruistic behaviours in animals and humans. An emerging field is evolutionary game theory played on graph structures. The purpose is to allow agents to interact only with their neighbours, and observe local phenomena. For example, pot-luck parties may be more successful if the group is small, or subgroups within the group do many pot-luck events. Because humans interact through a social network, it is important to also model games played on networks. In this thesis, random graphs are used as models for social networks, and the game is played on these random graphs under evolutionary dynamics, to explore how the structure of the random graph model influences strategy choice. 

\section{Summary}

Chapter \ref{Chapter:Lit} investigates the background and literature of evolutionary game theory. It is necessary to start with classical game theory. Historical background, as well as example games, are discussed to motivate evolutionary game theory. Evolutionary game theory is discussed and then expanded to include a graph structure. The current forefront of game theory literature is surveyed at the end of Chapter \ref{Chapter:Lit}. Also in Chapter \ref{Chapter:Lit}, there is a discussion on random graphs as models for social networks. This discussion informs the choice of random graph models for simulation. \\

From the literature, one particular paper is chosen for replication in Chapter \ref{TA}. This involves an update dynamic determined by payoff satisfaction. The paper is successfully replicated, and then the results are examined for robustness. Finally, four graph models are chosen from the literature, and the graph models explored under payoff satisfaction. Each graph model is controlled to have the same size and mean degree. It is found that the clustering coefficient informs the observed level of contribution. \\

Chapters \ref{Chapter:Rep} and \ref{Chapter:ID} both consider evolutionary update dynamics, specifically replicator and imitation dynamics. To provide authority to the thesis, another paper, instead playing a lottery game, using these dynamics is replicated, and then the dynamics are transposed to the public good game. Under both replicator and imitation dynamics, it is found that graph models with high variance of degree size induce higher contribution. This confirms results observed in the literature. Further tests are carried out in these chapters to confirm and refine this observation. The higher mean contribution is due to a high proportion of low-degree nodes supporting contribution. Due to the difference in replicator and imitation dynamics, replicator dynamics produces a much higher equilibrium contribution level, but takes longer to achieve this equilibrium. An explanation is provided. \\

This thesis confirms by simulation some results observed in the literature, and contains a novel exploration of the differences between replicator and imitation dynamics for a linear public goods game. Further study could involve the effect of modifying graph size, as well as investigating substructures of the graph to explain the observed results. \\
