\section{Local Imitation Dynamics for Public Good Games}
\subsection{Outline}
This chapter investigates the linear PGG under imitation dynamics. Imitation dynamics come from the same family as replicator dynamics introduced in \eqref{rep}, with $\alpha = 0$. Hence $q_{i\to j} = \mathbbm{1}_{\{\pi_j>\pi_i\}}$. Several differences are noticed between imitation and replicator dynamics. \\

\subsection{Results}
\FloatBarrier
\graphCap{ID_gtype_low.pdf}{0.7}{Comparing Graph Models: Imitation Dynamics. Trend for $r \in \{4.0, 4.25, 4.5, 4.75\}$. In each graph, the blue circles, orange stars, green crosses, and red pentagons correspond to the WS model, TAG model, BA model, and RRG model respectively.Only the BA graph model induces non-trivial cooperation.}{imitation_low} 
\FloatBarrier
\graphCap{ID_gtype_med.pdf}{0.7}{Comparing Graph Models: Imitation Dynamics. Trend for $r \in \{5.0, 5.25, 5.5, 5.75\}$. In each graph, the blue circles, orange stars, green crosses, and red pentagons correspond to the WS model, TAG model, BA model, and RRG model respectively. The BA model still induces the highest cooperation, and the RRG consistently induces the lowest.}{imitation_medium}
\FloatBarrier
\graphCap{ID_gtype_high.pdf}{0.7}{Comparing Graph Models: Replicator Dynamics. Trend for $r \in \{6,6.5,7,7.5\}$. In each graph, the blue circles, orange stars, green crosses, and red pentagons correspond to the WS model, TAG model, BA model, and RRG model respectively. The TAG model overtakes the BA model, while the RRG still fails to achieve non-zero cooperation. }{imitation_high}\FloatBarrier

Relative to replicator dynamics, less cooperation is observed for each graph model and parameter $r$. Furthermore, the \emph{re-ordering} effect noticed in Figure \ref{replicator_medium} and Figure \ref{replicator_high}, whereby the cooperation initially drops and then returns, is not observed here. \\

For $r<6.25$, the order of the models, from most to least cooperation, is the BA model, then TAG, then WS, and finally the RRG. In fact, for $r<7$, the RRG model never achieves non-zero cooperation. A possible explanation is the faster switching rate of strategies. Under imitation dynamics, player $i$ randomly chooses a neighbour $j$, and then always switches if $\pi_j>\pi_i$. In comparison, under replicator dynamics, the probability of switching is proportional to the difference in payoff. Therefore $i$ does not always switch to $j$, even if $\pi_j>\pi_i$. For this reason, the environment may be able to \emph{re-order}, and spread cooperation further before it becomes extinct. \\

In the regime $r\geq 6.25$, the TAG model overtakes the BA model for higher cooperation. This phenomenon was also noticed under replicator dynamics in Figure \ref{replicator_medium}, but instead occurred at $r=5.25$. \\

For consistency, the same tests from Chapter \ref{Rep} are reproduced under imitation dynamics. \\

\subsection{Isolating the Effect of Clustering Coefficient: Power Law Degree Size Distribution}

\FloatBarrier
\graphCap{ID_power_p_low.pdf}{0.7}{To test the effect of $C_\Delta$, the parameter $p$ in a PL graph was varied. In each graph, the blue circles, orange stars, green crosses, red pentagons, and purple squares correspond to $p\in\{0.1,0.2,0.3,0.4,0.5\}$ respectively. The observed trend is that higher $p$, and hence higher $C_\Delta$, leads to  higher cooperation.}{ID_power_p_low}
\FloatBarrier
\graphCap{ID_power_p_med.pdf}{0.7}{To test the effect of $C_\Delta$, the parameter $p$ in a PL graph was varied. In each graph, the blue circles, orange stars, green crosses, red pentagons, and purple squares correspond to $p \in \{0.1,0.2,0.3,0.4,0.5\}$ respectively. The trend is also observed here, with some minor discrepancies. For example, $p=0.4$ induces higher cooperation in a $r=5$ regime than $p=0.5$.}{ID_power_p_med} \FloatBarrier

The trend from Figure \ref{power_p_med} is reproduced here, namely that higher $p$, and hence higher $C_\Delta$, induces higher cooperation. This could be due to the higher clustering coefficient, or higher degree variance. Also note that the time to stability seems much faster in these models than Figures \ref{power_p_low}, \ref{power_p_med}, and \ref{power_p_high}. \\
\textbf{This trend is not as clear as in Rep, because I think ID induces higher rate of change, so more variance and flicking?. PL purple also does worse here. Seems faster to stability than Rep. Same overall trend as Rep}

\subsection{Isolating the Effect of Clustering Coefficient: Constant Degree Size Distribution}
\textbf{NEEDS Graphs here, }
\graphCap{ID_graph_p_med.pdf}{0.7}{Effect of Rewiring $p$ in a WS model. In each graph, the blue circles, orange stars, green crosses, red pentagons, and purple squares correspond to $p \in \{0.1,0.2,0.3,0.4,0.5\}$ respectively. The trend indicates that higher $p$, and hence lower $C_\Delta$, leads to higher cooperation. }{ID_graph_p_med}\FloatBarrier
 

\graphCap{ID_graph_p_high.pdf}{0.7}{Effect of Rewiring $p$ in a WS model. In each graph, the blue circles, orange stars, green crosses, red pentagons, and purple squares correspond to $p \in \{0.1,0.2,0.3,0.4,0.5\}$ respectively. The observed trend is an increase in $p$, and hence a reduction in $C_\Delta$, leads to higher cooperation.}{ID_graph_p_high}
\FloatBarrier


\subsection{The Effect of Mean Degree: RRG}

\graphCap{ID_graph_m_med_RRG.pdf}{0.8}{The effect of mean degree, RRG model. In each graph, the blue circles, orange stars, green crosses, red pentagons, and purple squares correspond to the mean degree $m \in \{4,6,8,10,12\}$ respectively. An increase in $m$ results in lower mean cooperation.  }{ID_graph_m_med} \FloatBarrier



\graphCap{ID_graph_m_high_RRG.pdf}{0.8}{The effect of mean degree, RRG model. In each graph, the blue circles, orange stars, green crosses, red pentagons, and purple squares correspond to mean degree $m \in  \{4,6,8,10,12\}$ respectively. Once again, an increase in $m$ results in lower cooperation }{ID_graph_m_high} \FloatBarrier

\textbf{Needs comment on RRG here, but pretty obvious that we will observe strict ESS, ie $r>m+1$ is necessary for $c=1>0$. Once again, way faster to equilibrium too. }
\FloatBarrier
\graphCap{ID_graph_m_low_BA.pdf}{0.8}{The effect of mean degree, BA model. In each graph, the blue circles, orange stars, green crosses, red pentagons, and purple squares correspond to targeted mean degree $m = \{4,6,8,10,12\}$ respectively. For the BA model, an increase in $m$ also results in lower cooperation.}{ID_BA_graph_m_low}
\FloatBarrier
\graphCap{ID_graph_m_med_BA.pdf}{0.8}{The effect of mean degree, BA model. In each graph, the blue circles, orange stars, green crosses, red pentagons, and purple squares correspond to targeted mean degree $m = \{4,6,8,10,12\}$ respectively. For the BA model, an increase in $m$ results in lower cooperation.}{ID_BA_graph_m_med}
\FloatBarrier
\textbf{Not as strict ESS in the BA model, but results look pretty similar to Rep. Do observe some non-extreme cooperation.}

\subsection{Cooperation Level Within a Single Graph Instance: BA model } 
\subsection{Differences Between Replicator and Imitation Dynamics}
\begin{itemize}
    \item Imitation dynamics moves a lot more often, as prob move given greater payoff is 1. This means goes to 0 faster, and potentially can't return
    \item Also means the diminishing return of defect in a defect environment is not punished
    \item Faster to equilibrium, as less \emph{waiting} for probabilities to hit
    \item Same ordering of graph models, but different switch time. I would hypothesise this is because it cares more about degree variance overall. 
    \item 
\end{itemize}
\subsection{Simple Results}
IN a fully connected graph, under imitation (and replicator) dynamics, then defection is an ESS regardless of r,m. \\
Proof. Simply note each agent is enrolled in all the games, so they each receive the split pot, but cooperators lose their input, so D>C, so in the long run we get all D. This is an ESS. \\

If $r>m+1$, then cooperating is a NE. However is is not observed, as there is no path to get there for a RRG. \\

Star graphs,

goes to 0 if C in the middle only.

goes to 0 if D in the middle and greater than 2 arms. 

So cannot ever have D in the middle. 

