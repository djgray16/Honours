\thispagestyle{plain}
\begin{center}
    \Large
    \textbf{Games on Random Graphs}
    
    \vspace{0.4cm}
    \large
    
    
    \vspace{0.4cm}
    \textbf{Daniel Gray}
    
    \vspace{0.9cm}
    \textbf{Abstract}
\end{center}
Evolutionary Game Theory is a model of interaction between competitive agents, whereby their strategies evolve over time to stabilise at an equilibrium level. In this thesis, agents are situated on a graph and interact with their neighbours in a linear public goods game (PGG). In a linear PGG,  each agent  chooses whether to contribute, and the contributions are enhanced and shared equally amongst all participants. The mean contribution proportion is compared for the linear public good game played on four different models for random networks. It is found that models with a higher variance of degree size induce higher cooperation.