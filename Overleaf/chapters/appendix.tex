\subsubsection{Proof of Existence of Nash Equlibrium}

The modern proof of the existence of a NE in any $n$-player game relies on Brouwer's fixed point theorem, although Nash also published a paper with Kakutani's fixed point theorem. To prove the existence of a NE, first note that the space of mixed strategy profiles is compact, as it is the cross product of bounded compact sets, \\
\begin{align*}
    \{ \mathbf{s}, \forall i, \sum_{j \in S_i} \alpha_j^i = 1 \}
\end{align*}
Given a mixed strategy profile $\mathbf{s}$, then the payoff to player $i$ is \\
\begin{align*}
    \pi_i(\mathbf{s}) = \sum_{k = 1}^n \sum_{s_k^j \in S_j} \alpha_k^j \pi_i(\mathbf{s_k}) \text{Need thomas check}
\end{align*}
This is the expanded sum of playing every pure strategy combination, weighted by its probability. \\
We can also write the payoff if player $i$ were to play pure strategy $s \in S_i$. \\
\begin{align*}
    \pi_i((s,\mathbf{s^{-i}})) = \sum_{k = 1, k \neq i}^n \sum_{s_k^j \in S_j} \alpha_k^j \pi_i((s,\mathbf{s_k^{-i}}))
\end{align*}

Then for every pure strategy of player $i$, $s_j^i \in s_i$, denote the function $p_i(s_j^i,\mathbf(s)) = \pi_i((s_j^i,\mathbf{s^{-i}})) - \pi_i(\mathbf(s))$. This can be seen as the advantage of moving to pure strategy $s_j^i$ from the current allocation. Then we can write a continuous function $\Phi$ to shift player $i$'s allocation to improve their utility, \\
\begin{align*}
    \Phi(\mathbf{s}) = \mathbf{s`} 
\end{align*}
    
\begin{align*}
    s'_i(s_j^i) = \frac{s_i(s_j^i) + \max{(0,p_i(s_j^i,\mathbf{s^{-i}}) )} }{1 + \sum_{s_k^i \in S_i} \max{(0,p_i(s_k^i,\mathbf{s^{-i}}) )}}
\end{align*}

The function $\Phi$ is clearly continuous, and over the compact space Brouwer's Fixed Point Theorem ensures the existence of a stationary point $\mathbf{s^*}$. Therefore, \\
\begin{align*}
    \Phi(\mathbf{s^*}) = \mathbf{s^*}. \\
\end{align*}
For all players $i$, take the sum over their allocations, \\
\begin{align*}
     \sum_{s_j^i \in S_i} s_i^*(s_j^i) = \sum_{s_j^i \in S_i} \frac{s_i^*(s_j^i) + \max{(0,p_i(s_j^i,\mathbf{s^{*-i}}) )}}{1 + \sum_{s_k^i \in S_i} \max{(0,p_i(s_k^i,\mathbf{{s^{*-i}}) )}} } \implies p_i(s_j^i,\mathbf{s^{-i}}) = 0
\end{align*}
This is precisely the definition of a NE, and its existence is proven. \\

\subsection{Exponential Random Graph Model}
 The exponential random graph model, alternatively called the $p^*$ model, was coined by Wasserman and Pattison, in honour of Leinhardt and Holland, who developed a model with one explanatory variable, a $p_1$ model \cite {RN83}. The $p^*$ model conceptualises the observed network $\by$ as one realisation of a set of possible networks generated by an underlying random process that depends on explanatory variables $\bz(\by) = [z_1(\by), z_2(\by), \dots, z_r(\by)]$ \cite{RN64}. The explanatory variables may be graph-theoretic measurements, such as the total number of edges, triangle proportion, or they may be external measurements of the actors at nodes \cite{RN86}. By assumption, the number of nodes is fixed, so it is not a parameter \cite{RN64}. Akin to generalised linear models \cite{RN83}, the underlying stochastic process is controlled by parameters $\bp = [p_1,p_2,\dots,p_r]$ that are coefficients of the linear function $$\sum_{i=1}^r z_i(\by)p_i. $$ 

 
 Assume that each edge $Y_{ij}$ is a random variable. The edges can be written into an $n\times n$ random matrix, denoted $\bY$. A particular realisation is denoted $\by$. Then the probability of observing the network $\by$ can be written as \\
 $$ \Prob (\bY = \by) = \frac{1}{D}\exp{\sum_{i=1}^r z_i(\by)p_i},$$
 
 where $D$ is a normalising constant. To avoid the intractability of $D$, the log odds are considered and conditioned on all other observed edges $y_{ij}^c$. This is the pseudo-log-likelihood. Write 
 $$\omega_{ij} = \log{\frac{\Prob (Y_{ij} =1 | y_{ij}^c)}{\Prob (Y_{ij} =0 | y_{ij}^c)}}.$$ 
 
$p^*$ models assume that these log-odds are conditionally independent. Denote $y_{ij}^+$ and $y_{ij}^-$ as the observed networks when edge $(i,j)$ is forced to be 1 and 0 respectively. Then the explanatory variables of the network are denoted $\bz(y_{ij}^+), \bz(y_{ij}^-) $ when $(i,j)$ is fixed at 1 and 0 respectively. This is simply computing the value of the explanatory variables for both cases of edge $(i,j)$. Then the pseudo-log-likelihood can be written $$
 \omega_{ij} = \exp \big (\bp \cdot [\bz(y_{ij}^+) - \bz(y_{ij}^-)] \big).$$
This is computationally tractable, and $\bp$ minimises the product of the $\omega_{ij}$ over every possible link in the graph. Actual solution techniques are discussed in Wasserman and Pattison's 1997 paper, but were found by Strauss (1986) and Strauss and Ikeda (1990) \cite{RN83}. Of note, Monte Carlo methods are demonstrably useful for parameter estimation \cite{RN86}. \\

A common explanatory variable is the triangle count, $\tau = \sum_{i,j,k} X_{i,j}X_{j,k}X_{k,i}$, so a $p^*$ model of a social network generally has a triangle parameter $p_\tau$ \cite{RN84}. Supposing that the researchers assumed the age of the actor has an effect as well, the simplest modelling technique is to introduce a different triangle parameter for each age class $(p_{\tau_{a_1}}, \dots, p_{\tau_{a_k}})$ for $k$ different age classes \cite{RN64}. Once a model has been developed, random graphs can be sampled from the specified distribution $\bp$ for analysis. For agent-based modelling, creating random graphs from a distribution with known parameters $\bp$ is useful to test how agent-based phenomenon interact on different graphs \cite{RN86}. \\

This model for networks has boundless potential for parametrisation. However there are two more common base models for social networks. The first is small-world networks, which seek to rectify the low clustering coefficient of $G(n,p)$ graphs to more accurately model social networks. \\
 
