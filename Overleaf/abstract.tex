\thispagestyle{plain}
\begin{center}
    \Large
    \textbf{Evolutionary Game Theory on Random Graphs}
    
    \vspace{0.4cm}
    \large
    
    
    \vspace{0.4cm}
    \textbf{Daniel Gray}
    
    \vspace{0.9cm}
    \textbf{Abstract}
\end{center}
Evolutionary Game Theory is a model of interaction between competitive agents, whereby their strategies evolve over time to stabilise at an equilibrium level. In this thesis, agents are situated on a graph and interact with their neighbours in a linear public goods game (PGG). In a linear PGG,  each agent  chooses whether to contribute, and the contributions are enhanced and shared equally amongst all participants. Agents then evolve their strategy according to a local update dynamic. The mean contribution proportion is compared for the linear public good game played on four different models for random networks. It is found that models with a higher variance of degree size induce higher cooperation. This is a result that corroborates the literature. Furthermore, the similarities and differences between two evolutionary update dynamics are investigated. Although they produce generally similar results, the differences are investigated and explained. 